%
% Bachelor / Masterarbeit mit LaTeX
% ===========================================================================
% This is part of the book "Diplomarbeit mit LaTeX".
% Copyright (c) 2002, 2003, 2005, 2007, 2008 Tobias Erbsland
% Copyright (c) 2005, 2006 Andreas Nitsch
% See the file main.tex for copying conditions.
%
% A. DOKUMENTKLASSE
% --------------------------------------------------------------------------

%  1. Definieren der Dokumentklasse.
%     Wir verwenden die KOMA-Script Klasse 'scrbook' für ein Buch.
%
\documentclass[%
    pdftex,%              PDFTex verwenden da wir ausschliesslich ein PDF erzeugen.
    a4paper,%             Wir verwenden A4 Papier.
    twoside,%             zweiseitiger Druck.
 %  openright,
 %  cleardoubleempty,
 %  BCOR12mm,
    12pt,%                Grosse Schrift, besser geeignet für A4.
    parskip=half,%         Halbe Zeile Abstand zwischen Absätzen.
    %chapterprefix,%       Kapitel mit 'Kapitel' anschreiben.
    headsepline,%         Linie nach Kopfzeile.
    %footsepline,%         Linie vor Fusszeile.
    bibliography=totoc,%  
    listof=totoc,%  Literaturverzeichnis im Inhaltsverzeichnis nummeriert einfügen.
    index=totoc%             Index ins Inhaltsverzeichnis einfügen.
]
{scrreprt}

\usepackage[a4paper,% wird eigentlich schon von der Klasse vorgegeben
  head=50mm,headsep=12mm,height=212mm,bottom=45mm]{geometry}

%\setlength{\textwidth}     {15.9cm}    % Textbreite
%\setlength{\oddsidemargin} {0cm}  % linker Rand (2.45-2.54) doppelseitig
%\setlength{\evensidemargin}{0cm}
%\setlength{\paperwidth}    {21.1cm}  % physikalische Seitenbreite
%\setlength{\textheight}    {23.5cm}  % Gesamthöhe für den Seitentext
%\setlength{\topmargin}     {-1cm}    % oberer Rand bis Oberkante Kopfzeile (1.5-2.54)

%
%  2. Festlegen der Zeichencodierung des Dokuments und des Zeichensatzes.
%     Wir verwenden 'UTF-8' für das Dokument,
%     und die 'T1' Codierung für die Schrift.
%
\usepackage[utf8]{inputenc}
\usepackage[T1]{fontenc}

%\usepackage{subfig}
\usepackage{caption}
\usepackage{subcaption}
\usepackage{placeins} 

%
%  3. Packet für die Index-Erstellung laden.
%
\usepackage{makeidx}

%
%  4. Paket für die Lokalisierung ins Deutsche laden.
%     Wir verwenden neue deutsche Rechtschreibung mit 'ngerman'.
%
\usepackage[english]{babel}


%
%  5. Paket für Anführungszeichen laden.
%     Wir setzen den Stil auf 'swiss', und verwenden so die Schweizer
%  Anführungszeichen.
%
\usepackage[style=swiss]{csquotes}


%
%  6. Paket für erweiterte Tabelleneigenschaften.
%
\usepackage{array}
\usepackage{bbm}

%
%  7. Paket um Grafiken im Dokument einbetten zu können.
%     Im PDF sind GIF, PNG, und PDF Grafiken möglich.
%
\usepackage{graphicx}
%
%  8. Pakete für mathematischen Textsatz.
%
\usepackage{amsmath}
\usepackage{amssymb}
\usepackage{dsfont}
\usepackage{mathtools}
%\usepackage{sty/algo}

%
%  9. Paket um Textteile drehen zu können.
%
\usepackage{rotating}

%
% 10. Paket für Farben an verschieden Stellen.
%     Wir definieren zusätzliche benannte Farben.
%
\usepackage{color}

%\usepackage[Lenny]{sty/fncychapleo}
%\usepackage[Glen]{sty/fncychapleo}


%
% 11. Paket für spezielle PDF features.
%
\usepackage[%
    pdftitle={Masterarbeit},%                        Titel des PDF Dokuments.
    pdfauthor={Max Mustermann},%              Autor des PDF Dokuments.
    pdfsubject={Titel der Masterarbeit/Bachelorarbeit},%                    Thema des PDF Dokuments.
    pdfcreator={MiKTeX, LaTeX with hyperref and KOMA-Script},% Erzeuger des PDF Dokuments.
    pdfkeywords={Masterarbeit,Bachelorarbeit,IES},%          Schlüsselwörter für das PDF.(Diese werden von Suchmaschinen auch für PDF Dokumente indexiert.)
    pdfpagemode=UseOutlines,%                                  Inhaltsverzeichnis anzeigen beim Öffnen
    pdfdisplaydoctitle=true,%                                  Dokumenttitel statt Dateiname anzeigen.
    pdflang=de,%  Sprache des Dokuments.
    pdfborder={0 0 0},
]{hyperref}


%
% 12. Paket um Quellcode sauber zu formatieren.
%     Mit der option 'savemem' verschieben wir das laden von
%     einzelnen Programmiersprachen auf einen späteren Zeitpunkt.
%
\usepackage[savemem]{listings}

%
% 13a. Privates Paket für die Schriftart 'Goudy Sans' laden.
%      Dieses Paket ist nur für die publizierte Version des Dokuments gedacht
%      und an dieser Stelle mit den nachfolgenden Anweisungen auskommentiert.
%
%\usepackage{goudysans}

%
% 13a. Font 'Latin Modern Family' verwenden.
%      Verwende dieses Paket wenn du DML selbst kompilierst.
%
\usepackage{lmodern}

%\usepackage{tocbasic}

%
% 15. Für Eingeben von Pseudocode
%

\usepackage[german,ruled,vlined]{algorithm2e}
\usepackage{algorithmic}
\usepackage{multirow}
\usepackage{tabularx,calc}

\usepackage{array} 
% B. EINSTELLUNGEN

% ---------------------------------------------------------------------------
%
%  1. Definieren von eigenen benannten Farben.
%     Für spätere Verwendung in dem Dokument, definieren wir einzelne
%     benannte Farben.
%
\definecolor{LinkColor}{rgb}{0,0,0}
\definecolor{CiteColor}{rgb}{0,0,0}
\definecolor{FileColor}{rgb}{0,0,0}
\definecolor{MenuColor}{rgb}{0,0,0}
\definecolor{UrlColor}{rgb}{0,0,0}
\definecolor{ListingBackground}{rgb}{0.85,0.85,0.85}
%
%  2. KOMA-Script Option, Zeilenumbruch bei Bildbeschreibungen.
%
\setcapindent{1em}

%
%  3. Stil der Kopf- und Fusszeilen.
%     Wir aktivieren mit 'headings' laufende Seitentitel.
%
\pagestyle{headings}

%
%  4. Stil der Überschriften auf normale Schrift.
%     Wir verwenden für die Überschriften den selben Font wie für den Text.
%
\setkomafont{sectioning}{\normalfont\bfseries}       % Titel mit Normalschrift
\setkomafont{captionlabel}{\normalfont\bfseries}     % Fette Beschriftungen
\setkomafont{pagehead}{\normalfont\itshape}          % Kursive Seitentitel
\setkomafont{descriptionlabel}{\normalfont\bfseries} % Fette Beschreibungstitel

%
%  5. Farbeinstellungen für die Links im PDF Dokument.
%
\hypersetup{%
    colorlinks=true,%        Aktivieren von farbigen Links im Dokument (keine Rahmen)
    linkcolor=LinkColor,%    Farbe festlegen.
    citecolor=CiteColor,%    Farbe festlegen.
    filecolor=FileColor,%    Farbe festlegen.
    menucolor=MenuColor,%    Farbe festlegen.
    urlcolor=UrlColor,%     Farbe von URL's im Dokument.
    bookmarksnumbered=true%  Überschriftsnummerierung im PDF Inhalt anzeigen.
}
%
%  6. Einstellungen für das 'listings' Paket.
%
\lstloadlanguages{TeX} % TeX sprache laden, notwendig wegen option 'savemem'
\lstset{%
    language=[LaTeX]TeX,     % Sprache des Quellcodes ist TeX
    numbers=left,            % Zelennummern links
    stepnumber=1,            % Jede Zeile nummerieren.
    numbersep=5pt,           % 5pt Abstand zum Quellcode
    numberstyle=\tiny,       % Zeichengrösse 'tiny' für die Nummern.
    breaklines=true,         % Zeilen umbrechen wenn notwendig.
    breakautoindent=true,    % Nach dem Zeilenumbruch Zeile einrücken.
    postbreak=\space,        % Bei Leerzeichen umbrechen.
    tabsize=2,               % Tabulatorgrösse 2
    basicstyle=\ttfamily\footnotesize, % Nichtproportionale Schrift, klein für den Quellcode
    showspaces=false,        % Leerzeichen nicht anzeigen.
    showstringspaces=false,  % Leerzeichen auch in Strings ('') nicht anzeigen.
    extendedchars=true,      % Alle Zeichen vom Latin1 Zeichensatz anzeigen.
    backgroundcolor=\color{ListingBackground}} % Hintergrundfarbe des Quellcodes setzen.

%
% C. NEUE MAKROS UND UMGEBUNGEN
% ---------------------------------------------------------------------------

\setcounter{secnumdepth}{4}
\setcounter{tocdepth}{3}
%
%  1. Umgebung für Änerungsliste mit einem speziellen Aufzählungszeichen.
%
\newenvironment{ListChanges}%
    {\begin{list}{$\diamondsuit$}{}}%
    {\end{list}}

%
%  2. Ersatz für die \LaTeX und \TeX Befehle für korrekte Darstellung.
%     Wir verwenden die 'Latin Modern Family' ('lm') als Font, da diese im
%     vergleich zu 'Computer Modern' ('cm') auch PostScript Dateien
%     anbieten, was zu einer schöneren Darstellung im PDF führt.
%
\newcommand{\DMLLaTeX}{{\fontfamily{lmr}\selectfont\LaTeX}}
\newcommand{\DMLTeX}{{\fontfamily{lmr}\selectfont\TeX}}

\newcommand{\HRule}{\rule{\linewidth}{0.5mm}}

\newcommand\addToPageCenter[1]{%
			\AddToShipoutPicuture*{\AtPageCenter{%
			\makebox(0,0){\includegraphics%
			[width=0.9\paperwidth]{#1}}}}}
			
			%%%% Abkürzungen für Befehle %%%%%
\newcommand{\bi}{\begin{itemize}}
\newcommand{\ei}{\end{itemize}}
\newcommand{\bc}{\begin{center}}
\newcommand{\ec}{\end{center}}
\newcommand{\bt}{\begin{tabular}}
\newcommand{\et}{\end{tabular}}
\newcommand{\ba}{\begin{array}}
\newcommand{\ea}{\end{array}}
\newcommand{\be}{\begin{enumerate}}
\newcommand{\ee}{\end{enumerate}}
\newcommand{\balg}{\begin{algorithm}}
\newcommand{\ealg}{\end{algorithm}}
\newcommand{\abs}[1]{\left\vert#1\right\vert}
\newcommand{\norm}[1]{\left\Vert#1\right\Vert}
\newcommand{\scalarproduct}[2]{\left<#1\vphantom{#2} \right| \! %
                                       \left.\vphantom{#1}#2\right>}
\newcommand{\scp}[2]{\scalarproduct{#1}{#2}}


\def\AmS{$\mathcal{A}$\kern-.1667em\lower.5ex\hbox
    {$\mathcal{M}$}\kern-.125em$\mathcal{S}$}
\def\AmSmath{\AmS{}math}
\usepackage{textcomp} 

\newcommand{\kommentar}[1]{}


%
% D. AKTIONEN
% ---------------------------------------------------------------------------
\newcommand*\restoreclearpage{}
\newcommand*\restorecleardoublepage{}
 \renewcommand*{\partheadendvskip}{%
 %\vfil\newpage
	\vspace{2cm}
  \expandafter\gdef\expandafter\restoreclearpage
  \expandafter{\expandafter\gdef\expandafter\clearpage
  \expandafter{\clearpage}}%
  \expandafter\gdef\expandafter\restorecleardoublepage
  \expandafter{\expandafter\gdef\expandafter\cleardoublepage
  \expandafter{\cleardoublepage}}%
  \gdef\clearpage{\restoreclearpage\restorecleardoublepage}%
  \global\let\cleardoublepage\clearpage
}

\newcommand{\changefont}[3]{
\fontfamily{#1} \fontseries{#2} \fontshape{#3} \selectfont}

\DeclareMathOperator*{\argmin}{argmin}
\DeclareMathOperator*{\argmax}{argmax}
\DeclareMathOperator*{\sign}{sgn}

%  1. Index erzeugen.
%
\makeindex
\pdfminorversion=6
%
% E. SILBENTRENNUNG
% ---------------------------------------------------------------------------
%
% Beispiele für Silbentrennungen

\hyphenation{Sach-bü-cher Mas-ter-ar-beit}

%
% ===========================================================================
% EOF
%